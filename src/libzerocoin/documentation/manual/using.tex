\section{Using libzerocoin}

The \libzerocoin library is designed to integrate with a Bitcoin/Litecoin style client, and performs the base cryptographic operations necessary to integrate Zerocoin with the client. These operations include generation/verification of coins, as well as generation/verification of spend signatures. Roughly speaking, the use of Zerocoin proceeds according to the following steps:

\begin{enumerate}
\item {\bf Parameter setup.} All Zerocoin clients in a deployment must share a single parameter $N$ where $N$ is a 2048-3072 bit modulus such that $N = p*q$ where $p$ and $q$ are large safe prime numbers (i.e., $p = 2p'+1$, $q = 2q'+1$ for primes $p', q'$). Once $N$ has been generated, the underlying values $p, q, p', q'$ can and should be destroyed.
 
In addition to $N$, all clients must agree on a security level $k$ (an integer $\ge 80$), as well as a canonical value of one zerocoin (measured in the underlying currency).

\item {\bf Coin generation.} To Mint a zerocoin, a client first generates a new coin $c$ using operations in the \libzerocoin~library. 

Once the coin is Minted, the client must now format and transmit a \textsf{ZEROCOIN\_MINT} transaction to the network, using routines not present in \libzerocoin. This transaction is similar to a normal Bitcoin/Litecoin transaction: it consists of inputs combining to the value of one zerocoin. Unlike a standard transaction, this transaction does not provide any outputs. Instead it simply embeds the Zerocoin value $c$. 
 
\end{enumerate} 