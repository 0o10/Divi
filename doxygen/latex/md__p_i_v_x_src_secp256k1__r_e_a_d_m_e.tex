\href{https://travis-ci.org/bitcoin/secp256k1}{\tt }

Optimized C library for EC operations on curve secp256k1.

This library is experimental, so use at your own risk.

Features\+:
\begin{DoxyItemize}
\item Low-\/level field and group operations on secp256k1.
\item E\+C\+D\+SA signing/verification and key generation.
\item Adding/multiplying private/public keys.
\item Serialization/parsing of private keys, public keys, signatures.
\item Very efficient implementation.
\end{DoxyItemize}

\subsection*{Implementation details }


\begin{DoxyItemize}
\item General
\begin{DoxyItemize}
\item Avoid dynamic memory usage almost everywhere.
\end{DoxyItemize}
\item Field operations
\begin{DoxyItemize}
\item Optimized implementation of arithmetic modulo the curve\textquotesingle{}s field size (2$^\wedge$256 -\/ 0x1000003\+D1).
\begin{DoxyItemize}
\item Using 5 52-\/bit limbs (including hand-\/optimized assembly for x86\+\_\+64, by Diederik Huys).
\item Using 10 26-\/bit limbs.
\item Using G\+MP.
\end{DoxyItemize}
\item Field inverses and square roots using a sliding window over blocks of 1s (by Peter Dettman).
\end{DoxyItemize}
\item Scalar operations
\begin{DoxyItemize}
\item Optimized implementation without data-\/dependent branches of arithmetic modulo the curve\textquotesingle{}s order.
\begin{DoxyItemize}
\item Using 4 64-\/bit limbs (relying on \+\_\+\+\_\+int128 support in the compiler).
\item Using 8 32-\/bit limbs.
\end{DoxyItemize}
\end{DoxyItemize}
\item Group operations
\begin{DoxyItemize}
\item Point addition formula specifically simplified for the curve equation (y$^\wedge$2 = x$^\wedge$3 + 7).
\item Use addition between points in Jacobian and affine coordinates where possible.
\item Use a unified addition/doubling formula where necessary to avoid data-\/dependent branches.
\end{DoxyItemize}
\item Point multiplication for verification (a$\ast$P + b$\ast$G).
\begin{DoxyItemize}
\item Use w\+N\+AF notation for point multiplicands.
\item Use a much larger window for multiples of G, using precomputed multiples.
\item Use Shamir\textquotesingle{}s trick to do the multiplication with the public key and the generator simultaneously.
\item Optionally use secp256k1\textquotesingle{}s efficiently-\/computable endomorphism to split the multiplicands into 4 half-\/sized ones first.
\end{DoxyItemize}
\item Point multiplication for signing
\begin{DoxyItemize}
\item Use a precomputed table of multiples of powers of 16 multiplied with the generator, so general multiplication becomes a series of additions.
\item Slice the precomputed table in memory per byte, so memory access to the table becomes uniform.
\item No data-\/dependent branches
\item The precomputed tables add and eventually subtract points for which no known scalar (private key) is known, preventing even an attacker with control over the private key used to control the data internally.
\end{DoxyItemize}
\end{DoxyItemize}

\subsection*{Build steps }

libsecp256k1 is built using autotools\+: \begin{DoxyVerb}$ ./autogen.sh
$ ./configure
$ make
$ sudo make install  # optional\end{DoxyVerb}
 